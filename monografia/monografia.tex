\documentclass{sigchi}

% Use this section to set the ACM copyright statement (e.g. for
% preprints).  Consult the conference website for the camera-ready
% copyright statement.

% Copyright
\CopyrightYear{2016}
%\setcopyright{acmcopyright}
\setcopyright{acmlicensed}
%\setcopyright{rightsretained}
%\setcopyright{usgov}
%\setcopyright{usgovmixed}
%\setcopyright{cagov}
%\setcopyright{cagovmixed}
% DOI
\doi{http://dx.doi.org/10.475/123_4}
% ISBN
\isbn{123-4567-24-567/08/06}
%Conference
\conferenceinfo{CHI'16,}{May 07--12, 2016, San Jose, CA, USA}
%Price
\acmPrice{\$15.00}

% Use this command to override the default ACM copyright statement
% (e.g. for preprints).  Consult the conference website for the
% camera-ready copyright statement.

%% HOW TO OVERRIDE THE DEFAULT COPYRIGHT STRIP --
%% Please note you need to make sure the copy for your specific
%% license is used here!
 \toappear{
 Permission to make digital or hard copies of all or part of this work
 for personal or classroom use is granted without fee provided that
 copies are not made or distributed for profit or commercial advantage
 and that copies bear this notice and the full citation on the first
 page. Copyrights for components of this work owned by others than ACM
 must be honored. Abstracting with credit is permitted. To copy
 otherwise, or republish, to post on servers or to redistribute to
 lists, requires prior specific permission and/or a fee. Request
 permissions from \href{mailto:Permissions@acm.org}{Permissions@acm.org}. \\
 \emph{CHI '16},  May 07--12, 2016, San Jose, CA, USA \\
 ACM xxx-x-xxxx-xxxx-x/xx/xx\ldots \$15.00 \\
 DOI: \url{http://dx.doi.org/xx.xxxx/xxxxxxx.xxxxxxx}
 }

% Arabic page numbers for submission.  Remove this line to eliminate
% page numbers for the camera ready copy
% \pagenumbering{arabic}

% Load basic packages
\usepackage[brazil]{babel}
\usepackage[utf8]{inputenc}
\usepackage{balance}       % to better equalize the last page
\usepackage{graphics}      % for EPS, load graphicx instead 
\usepackage[T1]{fontenc}   % for umlauts and other diaeresis
\usepackage{txfonts}
\usepackage{mathptmx}
\usepackage[pdflang={pt-BR},pdftex]{hyperref}
\usepackage{color}
\usepackage{booktabs}
\usepackage{textcomp}

% Some optional stuff you might like/need.
\usepackage{microtype}        % Improved Tracking and Kerning
% \usepackage[all]{hypcap}    % Fixes bug in hyperref caption linking
\usepackage{ccicons}          % Cite your images correctly!
% \usepackage[utf8]{inputenc} % for a UTF8 editor only

% If you want to use todo notes, marginpars etc. during creation of
% your draft document, you have to enable the "chi_draft" option for
% the document class. To do this, change the very first line to:
% "\documentclass[chi_draft]{sigchi}". You can then place todo notes
% by using the "\todo{...}"  command. Make sure to disable the draft
% option again before submitting your final document.
% \usepackage{todonotes}

% Paper metadata (use plain text, for PDF inclusion and later
% re-using, if desired).  Use \emtpyauthor when submitting for review
% so you remain anonymous.
\def\plaintitle{Melhorando a qualidade de decisões colaborativas para uma cidade inteligente}
\def\plainauthor{Carlos Elmadjian, Tarcisio Pereira}
\def\emptyauthor{}
\def\plainkeywords{escrever; aqui; obrigatório}
\def\plaingeneralterms{Documentation, Standardization}

% llt: Define a global style for URLs, rather that the default one
\makeatletter
\def\url@leostyle{%
  \@ifundefined{selectfont}{
    \def\UrlFont{\sf}
  }{
    \def\UrlFont{\small\bf\ttfamily}
  }}
\makeatother
\urlstyle{leo}

% To make various LaTeX processors do the right thing with page size.
\def\pprw{8.5in}
\def\pprh{11in}
\special{papersize=\pprw,\pprh}
\setlength{\paperwidth}{\pprw}
\setlength{\paperheight}{\pprh}
\setlength{\pdfpagewidth}{\pprw}
\setlength{\pdfpageheight}{\pprh}

% Make sure hyperref comes last of your loaded packages, to give it a
% fighting chance of not being over-written, since its job is to
% redefine many LaTeX commands.
\definecolor{linkColor}{RGB}{6,125,233}
\hypersetup{%
  pdftitle={\plaintitle},
% Use \plainauthor for final version.
%  pdfauthor={\plainauthor},
  pdfauthor={\emptyauthor},
  pdfkeywords={\plainkeywords},
  pdfdisplaydoctitle=true, % For Accessibility
  bookmarksnumbered,
  pdfstartview={FitH},
  colorlinks,
  citecolor=black,
  filecolor=black,
  linkcolor=black,
  urlcolor=linkColor,
  breaklinks=true,
  hypertexnames=false
}

% create a shortcut to typeset table headings
% \newcommand\tabhead[1]{\small\textbf{#1}}

% End of preamble. Here it comes the document.
\begin{document}

\title{\plaintitle}

\numberofauthors{2}
\author{%
  \alignauthor{Carlos Elmadjian\\
    \affaddr{IME / USP}\\
    \affaddr{São Paulo, Brasil}\\
    \email{elmad@ime.usp.br}}\\
  \alignauthor{Tarcisio Pereira\\
    \affaddr{IME / USP}\\
    \affaddr{São Paulo, Brasil}\\
    \email{tarcisio1@hotmail.com}}\\
}

\maketitle

\begin{abstract}
  resumo em 150 palavras.
\end{abstract}

\category{H.5.m.}{Information Interfaces and Presentation
  (e.g. HCI)}{Miscellaneous} \category{See
  \url{http://acm.org/about/class/1998/} for the full list of ACM
  classifiers. This section is required.}{}{}

\keywords{\plainkeywords}

\section{Introdução}
- introdução explicando a ideia do trabalho (e o título)\\
- descrição de trabalhos correlatos\\
- motivação e justificativas para o trabalho (referências resumidas na disciplina AQUI)\\


\begin{figure}
\centering
  \includegraphics[width=0.9\columnwidth]{figures/sigchi-logo}
  \caption{Insert a caption below each figure. Do not alter the
    Caption style.  One-line captions should be centered; multi-line
    should be justified. }~\label{fig:figure1}
\end{figure}


\begin{table}
  \centering
  \begin{tabular}{l r r r}
    % \toprule
    & & \multicolumn{2}{c}{\small{\textbf{Test Conditions}}} \\
    \cmidrule(r){3-4}
    {\small\textit{Name}}
    & {\small \textit{First}}
      & {\small \textit{Second}}
    & {\small \textit{Final}} \\
    \midrule
    Marsden & 223.0 & 44 & 432,321 \\
    Nass & 22.2 & 16 & 234,333 \\
    Borriello & 22.9 & 11 & 93,123 \\
    Karat & 34.9 & 2200 & 103,322 \\
    % \bottomrule
  \end{tabular}
  \caption{Table captions should be placed below the table. We
    recommend table lines be 1 point, 25\% black. Minimize use of
    table grid lines.}~\label{tab:table1}
\end{table}


\section{Ferramenta E-Voz}
- descrição do protótipo\\
- imagens com telas dos protótipos\\
- propósito do protótipo\\
- diferença para outras aplicações (estado da arte)\\

 %\begin{figure*}
 %  \centering
 %  \includegraphics[width=1.75\columnwidth]{figures/map}
 %  \caption{In this image, the map maximizes use of space. You can make
 %    figures as wide as you need, up to a maximum of the full width of
 %    both columns. Note that \LaTeX\ tends to render large figures on a
 %    dedicated page. Image: \ccbynd~ayman on
 %    Flickr.}~\label{fig:figure2}
 %\end{figure*}
 
 \section{Materiais e métodos}
 Os protótipos da ferramenta \textit{e-Voz} foram desenvolvidos utilizando as tecnologias HTML5, JavaScript e CSS3, sendo posteriormente compilados para a plataforma Android por meio do \textit{framework} Apache Cordova. Para o experimento com usuários, foi utilizado um aparelho Motorola Moto G Dual SIM, com uma tela de 4,5 polegadas.
 
 \subsection{Design experimental}
 Para a verificação da hipótese aventada na Introdução, propusemos um teste A/B com os participantes entre os dois protótipos investigados, de modo que a única diferença visual entre ambos era o botão adicional de recursos estatísticos, presente na segunda versão [FIGURA X]. Desse modo, procurou-se minimizar a possibilidade de que outras variáveis relativas ao experimento tivessem algum impacto deletério sobre os resultados.
 
 No teste proposto, os indivíduos deveriam completar a mesma tarefa com cada versão. Para reduzir vieses experimentais relativos ao aprendizado da tarefa, dividimos aleatoriamente os participantes em dois grupos: no primeiro, os indivíduos inciavam o experimento com o protótipo I e depois com o II, enquanto no segundo a ordem foi invertida. Ao final de cada tarefa, solicitava-se ao participante que respondesse a um questionário [TABELA X] com quatro questões cujas respostas foram dispostas em uma escala de Likert de cinco níveis~\cite{likert:1932}, com o intuito de verificar qual a experiência de usuário obtida.
 
 Optou-se por realizar uma investigação inteiramente intra-sujeito e sem comparação com outras ferramentas equivalentes, a fim eliminar o viés de confirmação tipicamente presente em experimentos dessa natureza~\cite{dell:2012}. Após a conclusão das tarefas, uma entrevista semiestruturada deveria se realizada com cada participante a fim de identificar o cumprimento ou não de critérios de usabilidade, registrar os sentimentos relatados, investigar eventuais mudanças de comportamento na interação com cada protótipo e coletar opiniões gerais sobre a ferramenta.
 
 \subsection{Protocolo experimental}
 Todos os participantes se sujeitaram ao experimento em ambientes naturais. Ao ser abordado, o voluntário recebia informações sobre natureza da investigação, a garantia do experimentador de que todas as informações colhidas seriam confidenciais e o compromisso de que ele poderia desistir do experimento a qualquer momento que desejasse.
 
 Uma vez de acordada sua participação, o experimentador exibia ao voluntário um dos protótipos da ferramenta \textit{e-Voz}, mostrando todas as suas telas e recursos interativos para que o participante pudesse se familiarizar com o aplicativo. Durante essa etapa, ele também era instruído sobre o contexto em que estava inserida a ferramenta e qual seria a natureza da tarefa requisitada, isto é, definir cinco problemas como prioritários dentre os 15 dispostos sobre o mapa do aplicativo com o intuito de auxiliar a Prefeitura de São Paulo na alocação de recursos públicos, utilizando ou não o recurso de auxílio estatístico (no caso da versão II).
 
 Enquanto o usuário selecionava os problemas que na sua opinião eram prioritários, o experimentador cronometrava o tempo gasto com a tarefa. Finda essa etapa, o participante recebia o questionário [XXX], para o qual lhe era informado explicitamente de que não havia tempo máximo para conclusão. Em seguida, a outra versão do protótipo lhe era apresentada, ressaltando a principal diferença em comparação com a anterior, e lhe era requisitado que selecionasse os cinco problemas prioritários. Novamente, a tarefa era cronometrada e, ao seu término, o mesmo questionário [XXX] era aplicado.
 
 Concluída essa etapa, o experimentador solicitava ao participante mais alguns minutos para responder perguntas sobre a sua experiência com os protótipos. Durante a entrevista, foram feitas algumas questões abertas pré-definidas ao usuário [TABELA XXXX], bem como outras perguntas que o experimentador julgasse pertinente no contexto, tanto para clarificar uma resposta dada quanto para dirimir dúvidas sobre eventuais incoerências entre o comportamento observado pelo experimentador e o relatado pelo participante.


\section{Resultados}
- perfil dos participantes (idade, residência, sexo)\\
- estatísticas das respostas\\
- análise estatística\\
- tabelas com resultados\\


\section{Discussão}
Os dados reforçam a hipótese do viés de disponibilidade~\cite{tversky:1973} na tomada de decisão dos participantes. No grupo que interagiu primeiro com o protótipo I e depois com o II, nota-se uma alteração no critério de escolha da prioridade dos problema com o aumento de informações que subsidiam a tarefa. O mesmo não se observa no grupo que inicia a interação pelo protótipo II e depois segue para o I. Quando questionados sobre a mudança de postura, a maior parte dos participantes reconheceu utilizar um critério individualista quando não havia informações estatísticas disponíveis sobre a cidade, o que parece indicar que o recurso impacta positivamente para tomada de decisões mais altruístas ou coletivistas.

\section{Limitações e trabalho futuro}
- quantidade de participantes\\
- utilizar meios e equipamentos informáticos faz que apenas parte da população, que tem acesso à essas tecnologias, possa contribuir, não abrangindo toda a população da cidade\\
- avaliação de outros fatures que contribuem para a interação colaborativa (rede social, por exemplo)\\
- avaliar se os recursos de gamification estimulam o usuário a usar mais a aplicação, e se em conjunto com a característica de rede social a aplicação continua sendo para resolver problemas da cidade ou se o propósito inicial se torna vago\\
- O quanto que essa mudança realmente impacta apenas será possível analisar com o tempo e a real implementação.

\section{Conclusão}
- oferecer recursos para melhora de decisão em plataformas colaborativas sem onerar a usabilidade de interfaces pode inicialmente ser um desafio, mas os efeitos aparentes tanto na qualidade da interação (percepção dos usuários) quanto na utilidade e resultados indicam um benefício significativo.


\section{Agradecimentos}

Agradecemos a todos os participantes que se submeteram voluntariamente ao experimento e aos nossos revisores, com seus comentários e críticas valiosas para o trabalho.


\balance{}


% BALANCE COLUMNS
\balance{}

% REFERENCES FORMAT
% References must be the same font size as other body text.
\bibliographystyle{SIGCHI-Reference-Format}
\bibliography{sample}

\end{document}

%%% Local Variables:
%%% mode: latex
%%% TeX-master: t
%%% End:
